% Options for packages loaded elsewhere
% Options for packages loaded elsewhere
\PassOptionsToPackage{unicode}{hyperref}
\PassOptionsToPackage{hyphens}{url}
\PassOptionsToPackage{dvipsnames,svgnames,x11names}{xcolor}
%
\documentclass[
  letterpaper,
  DIV=11,
  numbers=noendperiod]{scrreprt}
\usepackage{xcolor}
\usepackage{amsmath,amssymb}
\setcounter{secnumdepth}{5}
\usepackage{iftex}
\ifPDFTeX
  \usepackage[T1]{fontenc}
  \usepackage[utf8]{inputenc}
  \usepackage{textcomp} % provide euro and other symbols
\else % if luatex or xetex
  \usepackage{unicode-math} % this also loads fontspec
  \defaultfontfeatures{Scale=MatchLowercase}
  \defaultfontfeatures[\rmfamily]{Ligatures=TeX,Scale=1}
\fi
\usepackage{lmodern}
\ifPDFTeX\else
  % xetex/luatex font selection
\fi
% Use upquote if available, for straight quotes in verbatim environments
\IfFileExists{upquote.sty}{\usepackage{upquote}}{}
\IfFileExists{microtype.sty}{% use microtype if available
  \usepackage[]{microtype}
  \UseMicrotypeSet[protrusion]{basicmath} % disable protrusion for tt fonts
}{}
\makeatletter
\@ifundefined{KOMAClassName}{% if non-KOMA class
  \IfFileExists{parskip.sty}{%
    \usepackage{parskip}
  }{% else
    \setlength{\parindent}{0pt}
    \setlength{\parskip}{6pt plus 2pt minus 1pt}}
}{% if KOMA class
  \KOMAoptions{parskip=half}}
\makeatother
% Make \paragraph and \subparagraph free-standing
\makeatletter
\ifx\paragraph\undefined\else
  \let\oldparagraph\paragraph
  \renewcommand{\paragraph}{
    \@ifstar
      \xxxParagraphStar
      \xxxParagraphNoStar
  }
  \newcommand{\xxxParagraphStar}[1]{\oldparagraph*{#1}\mbox{}}
  \newcommand{\xxxParagraphNoStar}[1]{\oldparagraph{#1}\mbox{}}
\fi
\ifx\subparagraph\undefined\else
  \let\oldsubparagraph\subparagraph
  \renewcommand{\subparagraph}{
    \@ifstar
      \xxxSubParagraphStar
      \xxxSubParagraphNoStar
  }
  \newcommand{\xxxSubParagraphStar}[1]{\oldsubparagraph*{#1}\mbox{}}
  \newcommand{\xxxSubParagraphNoStar}[1]{\oldsubparagraph{#1}\mbox{}}
\fi
\makeatother


\usepackage{longtable,booktabs,array}
\usepackage{calc} % for calculating minipage widths
% Correct order of tables after \paragraph or \subparagraph
\usepackage{etoolbox}
\makeatletter
\patchcmd\longtable{\par}{\if@noskipsec\mbox{}\fi\par}{}{}
\makeatother
% Allow footnotes in longtable head/foot
\IfFileExists{footnotehyper.sty}{\usepackage{footnotehyper}}{\usepackage{footnote}}
\makesavenoteenv{longtable}
\usepackage{graphicx}
\makeatletter
\newsavebox\pandoc@box
\newcommand*\pandocbounded[1]{% scales image to fit in text height/width
  \sbox\pandoc@box{#1}%
  \Gscale@div\@tempa{\textheight}{\dimexpr\ht\pandoc@box+\dp\pandoc@box\relax}%
  \Gscale@div\@tempb{\linewidth}{\wd\pandoc@box}%
  \ifdim\@tempb\p@<\@tempa\p@\let\@tempa\@tempb\fi% select the smaller of both
  \ifdim\@tempa\p@<\p@\scalebox{\@tempa}{\usebox\pandoc@box}%
  \else\usebox{\pandoc@box}%
  \fi%
}
% Set default figure placement to htbp
\def\fps@figure{htbp}
\makeatother





\setlength{\emergencystretch}{3em} % prevent overfull lines

\providecommand{\tightlist}{%
  \setlength{\itemsep}{0pt}\setlength{\parskip}{0pt}}



 


\KOMAoption{captions}{tableheading}
\makeatletter
\@ifpackageloaded{bookmark}{}{\usepackage{bookmark}}
\makeatother
\makeatletter
\@ifpackageloaded{caption}{}{\usepackage{caption}}
\AtBeginDocument{%
\ifdefined\contentsname
  \renewcommand*\contentsname{Table of contents}
\else
  \newcommand\contentsname{Table of contents}
\fi
\ifdefined\listfigurename
  \renewcommand*\listfigurename{List of Figures}
\else
  \newcommand\listfigurename{List of Figures}
\fi
\ifdefined\listtablename
  \renewcommand*\listtablename{List of Tables}
\else
  \newcommand\listtablename{List of Tables}
\fi
\ifdefined\figurename
  \renewcommand*\figurename{Figure}
\else
  \newcommand\figurename{Figure}
\fi
\ifdefined\tablename
  \renewcommand*\tablename{Table}
\else
  \newcommand\tablename{Table}
\fi
}
\@ifpackageloaded{float}{}{\usepackage{float}}
\floatstyle{ruled}
\@ifundefined{c@chapter}{\newfloat{codelisting}{h}{lop}}{\newfloat{codelisting}{h}{lop}[chapter]}
\floatname{codelisting}{Listing}
\newcommand*\listoflistings{\listof{codelisting}{List of Listings}}
\makeatother
\makeatletter
\makeatother
\makeatletter
\@ifpackageloaded{caption}{}{\usepackage{caption}}
\@ifpackageloaded{subcaption}{}{\usepackage{subcaption}}
\makeatother
\usepackage{bookmark}
\IfFileExists{xurl.sty}{\usepackage{xurl}}{} % add URL line breaks if available
\urlstyle{same}
\hypersetup{
  pdftitle={All About Me},
  pdfauthor={131902360 Armein Z R Langi},
  colorlinks=true,
  linkcolor={blue},
  filecolor={Maroon},
  citecolor={Blue},
  urlcolor={Blue},
  pdfcreator={LaTeX via pandoc}}


\title{All About Me}
\usepackage{etoolbox}
\makeatletter
\providecommand{\subtitle}[1]{% add subtitle to \maketitle
  \apptocmd{\@title}{\par {\large #1 \par}}{}{}
}
\makeatother
\subtitle{Portfolio Asesmen II-2100 KIPP}
\author{131902360 Armein Z R Langi}
\date{2025-09-15}
\begin{document}
\maketitle

\renewcommand*\contentsname{Table of contents}
{
\hypersetup{linkcolor=}
\setcounter{tocdepth}{2}
\tableofcontents
}

\bookmarksetup{startatroot}

\chapter*{Selamat Berjumpa}\label{selamat-berjumpa}
\addcontentsline{toc}{chapter}{Selamat Berjumpa}

\markboth{Selamat Berjumpa}{Selamat Berjumpa}

\begin{figure}[H]

{\centering \includegraphics[width=9.5\linewidth,height=\textheight,keepaspectratio]{images/AZRL.png}

}

\caption{About Me}

\end{figure}%

Armein Z R Langi adalah Guru Besar di Sekolah Teknik Elektro dan
Informatika ITB, dosen ITB sejak Desember 1987, mantan Rektor
Universitas Kristen Maranatha, 1 Maret 2016 s/d 29 Februari 2020, mantan
Kepala Pusat Penelitian Teknologi Informasi dan Komunikasi (PP-TIK) ITB
November 2005 s/d Maret 2010, dan Sekretaris MWA ITB Mei 2010-Jan 2011.

Lahir di Tomohon 1962 dari pasangan Manado dan Sunda. Saat ini tinggal
di Bandung, menikah dengan Ina dan dikaruniai empat anak. Ayah dari
Gladys, Kezia, Andria, dan Marco.

Sharing pikiran singkat ada di blog \url{https://azrl.wordpress.com}.
Facebookk: armein\_langi

Andria, putri saya, lumayan sering membaca blog ini. Dia juga sumber
inspirasi tulisan saya. Dia senang menceritakan jokes pada saya, dan
kalau bagus saya tulis di sini, terutama seri Kocak Ala Andria. Dan dia
suka heran, karena isinya sudah berbeda. Memang saya senang
mengubah-ngubah cerita karena saya tidak suka menjiplak mentah-mentah.
Minggu lalu tidak sengaja dia memuji tulisan blog ini.

``Papa\ldots{} papa\ldots{} ``, tanyanya, ``Tulisan blog papa itu
copy-paste dari tulisan orang ya\ldots?''

Haa? Ya nggak mungkin lah. Kecuali lirik lagu, semua artikel di sini
ditulis sendiri.

``Ah bohong, soalnya tulisannya terlalu banyak\ldots.'' katanya tetap
tidak percaya, ``Nggak mungkinlah papa tulis sendiri\ldots{}''

Hehe, saya tersenyum sambil membelai Andria, karena buat saya itu adalah
ultimate compliment. Thanks sweetheart\ldots{}

Jadi kalau orang tidak percaya bahwa sesuatu itu karya anda, anda yang
buat, jangan marah. Itu adalah pujian yang sejati.

\bookmarksetup{startatroot}

\chapter{UTS-1 All About Me}\label{uts-1-all-about-me}

\section{Awal Perjalanan}\label{awal-perjalanan}

Halo, aku Lukman. Sejak kecil, aku selalu punya rasa ingin tahu yang
besar tentang banyak hal. Aku bukan tipe anak yang gampang puas dengan
satu jawaban, kalau ada sesuatu yang bikin penasaran, aku bakal cari
tahu sampai bener-bener paham. Kadang, keingintahuan itu bikin aku
terjun ke hal-hal yang di luar zona nyamanku. Tapi justru dari situ aku
belajar banyak hal, termasuk gimana rasanya jatuh, gagal, dan bangkit
lagi.

Aku tumbuh dengan banyak pengalaman yang pelan-pelan ngebentuk cara
pandangku terhadap dunia. Ada momen-momen kecil yang mungkin bagi orang
lain nggak signifikan, tapi buatku justru jadi fondasi yang penting. Aku
belajar untuk mandiri, untuk nggak selalu bergantung, dan untuk tetap
punya pegangan meskipun situasi berubah-ubah.

\section{Tentang Aku}\label{tentang-aku}

Aku orangnya cenderung reflektif dan gampang mikir hal-hal kecil. Kadang
aku bisa diem lama cuma buat mikirin satu kejadian, satu kalimat, atau
satu ekspresi orang. Aku nggak selalu terbuka ke semua orang, tapi kalau
udah ngerasa nyaman, aku bisa cerita panjang banget sampai lupa waktu.

Aku juga cukup perfeksionis dalam beberapa hal. Kalau ngerjain sesuatu,
aku pengen hasilnya maksimal, meskipun kadang bikin aku kelelahan
sendiri. Tapi di sisi lain, aku juga sadar kalau hidup bukan soal
kesempurnaan, tapi soal gimana kita bisa terus jalan meskipun pelan. Aku
belajar buat lebih nerima diriku sendiri, dengan segala kekurangan dan
kelebihannya.

\section{Perjalanan Akademik}\label{perjalanan-akademik}

Perjalanan akademikku nggak selalu lurus dan mulus. Ada masa-masa di
mana aku ngerasa capek banget, ngerasa stuck, dan ngeragukan diri
sendiri. Tapi ada juga masa-masa di mana aku ngerasa bangga karena
berhasil ngelewatin sesuatu yang dulu aku pikir mustahil. Kuliah bukan
cuma soal dapet nilai bagus, tapi juga soal proses pendewasaan.

Aku belajar untuk lebih disiplin, ngatur waktu, dan nggak gampang
nyerah. Banyak hal yang awalnya bikin aku stres, sekarang malah jadi
pelajaran berharga yang bikin aku lebih kuat. Setiap tugas, presentasi,
dan ujian yang aku hadapi bukan sekadar beban, tapi batu pijakan buat
langkahku ke depan. Walaupun kadang chaos, aku tetap berusaha nginget
alasan kenapa aku mulai, supaya semangatnya nggak hilang begitu aja.

\section{Hal-hal yang Aku Sukai}\label{hal-hal-yang-aku-sukai}

Di luar aktivitas akademik, aku punya banyak hal kecil yang bikin
hidupku jadi lebih ringan dan hangat. Aku suka dengerin musik
malam-malam, terutama saat semua mulai tenang. Rasanya kaya ngobrol sama
diri sendiri, tapi lewat nada dan lirik. Kadang aku juga suka nulis,
entah itu cerita, curhatan pribadi, atau sekadar catatan kecil tentang
apa yang aku rasain hari itu.

Aku suka ngamatin orang juga. Bukan dalam arti kepo, tapi lebih ke
menikmati cara setiap orang punya keunikan masing-masing. Ada hal
menarik dalam setiap gesture, cara ngomong, dan cara mereka menghadapi
hidup. Hal-hal kecil kayak gini yang bikin aku sadar kalau dunia itu
luas banget, dan setiap orang bawa ceritanya masing-masing.

\section{Refleksi Pribadi}\label{refleksi-pribadi}

Kalau aku ngeliat ke belakang, aku bersyukur banget sama semua proses
yang udah aku lewatin. Meski banyak jatuhnya, semua itu bikin aku lebih
peka, lebih kuat, dan lebih ngerti siapa diriku sebenarnya. Aku masih
jauh dari kata ``selesai'' dimana masih banyak hal yang pengen aku
capai, pelajari, dan perbaiki. Tapi satu hal yang pasti, aku nggak lagi
jalan cuma buat nyenengin orang lain. Sekarang aku jalan buat bahagiain
diriku sendiri.

Aku percaya setiap orang punya waktunya masing-masing. Aku nggak
buru-buru, tapi juga nggak berhenti. Aku pelan-pelan ngebangun hidup
yang aku pengen, dengan versiku sendiri.

\begin{center}\rule{0.5\linewidth}{0.5pt}\end{center}

\bookmarksetup{startatroot}

\chapter{UTS-2 My Songs for You}\label{uts-2-my-songs-for-you}

\bookmarksetup{startatroot}

\chapter{UTS-3 My Stories for You}\label{uts-3-my-stories-for-you}

\bookmarksetup{startatroot}

\chapter{UTS-4 My SHAPE (Spiritual Gifts, Heart, Abilities, Personality,
Experiences)}\label{uts-4-my-shape-spiritual-gifts-heart-abilities-personality-experiences}

\bookmarksetup{startatroot}

\chapter{UTS-5 My Personal Reviews}\label{uts-5-my-personal-reviews}

Berikut cara saya melakukan review: mengguan chatGPT, saya mengattach
\href{skor_uts.pdf}{file promt ChatGPT}, disertai perintah :``self
assess uts-1 sanpai uts-5 dari URL
`https://ii-2100.github.io/all-about-me/'\,''

ChatGPT melakukan self-assessment UTS-1 s.d. UTS-5 langsung dari laman
yang Anda berikan dan menilai memakai rubrik tugas UTS (skala 1--5 per
kriteria). Rekap skor siap diunduh sebagai CSV:
\href{sandbox:/mnt/data/UTS_self_assessment.csv}{Download CSV
ringkasan}.

\bookmarksetup{startatroot}

\chapter{Hasil Self-Assessment UTS (URL:
ii-2100.github.io/all-about-me)}\label{hasil-self-assessment-uts-url-ii-2100.github.ioall-about-me}

\section{Identifikasi}\label{identifikasi}

\begin{itemize}
\tightlist
\item
  Nama \& NIM penulis: \textbf{Armein Z. R. Langi -- 131902360} (tertera
  di halaman depan portofolio).
  (\href{https://ii-2100.github.io/all-about-me/}{II 2100})
\item
  Penilai: \textbf{Self-assessment (Armein Z. R. Langi)}
\item
  Catatan cakupan: halaman beranda memuat ``About Me''; navigasi ke ``My
  Songs for You'', ``My Stories for You'', ``My Shapes'', dan ``My
  Personal Reviews'' tersedia.
  (\href{https://ii-2100.github.io/all-about-me/}{II 2100})
\end{itemize}

\section{Tinjauan Umum}\label{tinjauan-umum}

\begin{itemize}
\tightlist
\item
  \textbf{UTS-1 (All About Me)} hadir di beranda (``Selamat Berjumpa /
  About Me''). Isi memperkenalkan identitas dan latar personal secara
  padat. (\href{https://ii-2100.github.io/all-about-me/}{II 2100})
\item
  \textbf{UTS-2 (My Songs for You)} memuat judul karya dan tautan audio,
  namun lirik/isi tidak ditampilkan di halaman (file audio tidak bisa
  saya akses dari sini), sehingga penilaian konten terbatas pada
  kelengkapan presentasi.
  (\href{https://ii-2100.github.io/all-about-me/My_Song_for_You/index.html}{II
  2100})
\item
  \textbf{UTS-3 (My Stories for You)} berisi tautan ke beberapa cerita;
  konten naratif kuat, terutama pada kisah keluarga ``Gaun Pengantin
  Gladys'' serta kisah inspiratif lainnya.
  (\href{https://ii-2100.github.io/all-about-me/My_Stories_for_You/index.html}{II
  2100})
\item
  \textbf{UTS-4 (My SHAPE)} saat ini halaman kosong/tanpa isi, hanya
  navigasi.
  (\href{https://ii-2100.github.io/all-about-me/My_Shapes/index.html}{II
  2100})
\item
  \textbf{UTS-5 (My Personal Reviews)} berisi metode/tautan panduan
  review, namun belum ada contoh \textbf{review personal} yang lengkap
  terhadap sebuah pesan/teks.
  (\href{https://ii-2100.github.io/all-about-me/My_Personal_Reviews/index.html}{II
  2100})
\end{itemize}

\begin{center}\rule{0.5\linewidth}{0.5pt}\end{center}

\section{Tinjauan Spesifik + Skor
(1--5)}\label{tinjauan-spesifik-skor-15}

\subsection{UTS-1 --- All About Me (di
beranda)}\label{uts-1-all-about-me-di-beranda}

\textbf{Skor per kriteria:} Orisinalitas \textbf{3}, Keterlibatan
\textbf{3}, Humor \textbf{1}, Wawasan/Insight \textbf{3} → \textbf{Total
10/20 (50\%)}. \textbf{Alasan singkat:} Perkenalan diri faktual dan
jelas, namun gaya masih informatif ketimbang naratif (minim humor dan
moment ``hook''). (\href{https://ii-2100.github.io/all-about-me/}{II
2100}) \textbf{Saran perbaikan:} Buka dengan anekdot pribadi (1--2
paragraf) yang ``mengikat'' (mis. titik balik karier/keluarga),
tambahkan satu momen humor ringan, lalu tutup dengan refleksi singkat
tentang nilai/visi diri agar aspek insight naik.

\subsection{UTS-2 --- My Songs for You}\label{uts-2-my-songs-for-you-1}

\textbf{Skor per kriteria:} Orisinalitas \textbf{2}, Keterlibatan
\textbf{2}, Humor \textbf{1}, Inspirasi \textbf{2} → \textbf{Total 7/20
(35\%)}. \textbf{Alasan singkat:} Halaman menampilkan judul lagu ``River
in my Mind'', ``Heaven on Earth'', namun tanpa lirik/cerita di balik
lagu sehingga sulit menilai aspek pesan, humor, dan inspirasi.
(\href{https://ii-2100.github.io/all-about-me/My_Song_for_You/index.html}{II
2100}) \textbf{Saran perbaikan:} Tambahkan lirik lengkap, 1 paragraf
cerita proses kreatif, dan 2--3 kalimat ``pesan untukmu'' agar inspirasi
terbaca; sertakan player/tautan yang dapat diputar langsung + fallback
transkrip.

\subsection{UTS-3 --- My Stories for
You}\label{uts-3-my-stories-for-you-1}

\textbf{Skor per kriteria:} Orisinalitas \textbf{5}, Keterlibatan
\textbf{5}, Pengembangan Narasi \textbf{4}, Inspirasi \textbf{5} →
\textbf{Total 19/20 (95\%)}. \textbf{Alasan singkat:} Cerita ``Gaun
Pengantin Gladys'' dkk sangat personal, emosional, dan inspiratif; ritme
bertutur hidup serta detail situasional kuat (konten ramu unsur
penebusan/keluarga/iman).
(\href{https://ii-2100.github.io/all-about-me/My_Stories_for_You/index.html}{II
2100}) \textbf{Saran perbaikan:} Tambah ``lead'' 2--3 kalimat yang
merangkum pesan kunci tiap cerita; akhiri dengan ajakan/refleksi 1--2
kalimat agar resonansi ke pembaca makin jelas.

\subsection{UTS-4 --- My SHAPE}\label{uts-4-my-shape}

\textbf{Skor per kriteria:} Orisinalitas \textbf{1}, Keterlibatan
\textbf{1}, Pengembangan \textbf{1}, Inspirasi \textbf{1} →
\textbf{Total 4/20 (20\%)}. \textbf{Alasan singkat:} Konten belum
tersedia.
(\href{https://ii-2100.github.io/all-about-me/My_Shapes/index.html}{II
2100}) \textbf{Saran perbaikan (prioritas):}

\begin{enumerate}
\def\labelenumi{\arabic{enumi}.}
\tightlist
\item
  Tampilkan rekap SHAPE (Spiritual gifts, Heart, Abilities, Personality,
  Experiences) dalam 1 tabel ringkas + 3--5 butir bukti;
\item
  Narasikan 2 pengalaman kunci (±150 kata/each) yang menunjukkan
  ``signature strengths'';
\item
  Tutup dengan 3 rencana aksi konkret (who/what/when).
\end{enumerate}

\subsection{UTS-5 --- My Personal
Reviews}\label{uts-5-my-personal-reviews-1}

\textbf{Skor per kriteria:} Pemahaman Konsep \textbf{2}, Analisis Kritis
\textbf{1}, Argumentasi (Logos) \textbf{1}, Etos \& Empati \textbf{2},
Rekomendasi \textbf{1} → \textbf{Total 7/25 (28\%)}. \textbf{Alasan
singkat:} Halaman berisi metode cara menilai, tetapi \textbf{belum ada}
contoh \textbf{review personal} yang lengkap terhadap sebuah pesan/teks
sehingga aspek analisis-argumentasi tak terbaca.
(\href{https://ii-2100.github.io/all-about-me/My_Personal_Reviews/index.html}{II
2100}) \textbf{Saran perbaikan:} Pilih 1 karya personal (mis.
UTS-1/2/3), tulis review 400--600 kata: ringkas pesan, nilai dengan
rubrik, berikan 2--3 bukti kutipan, evaluasi etos/empati, lalu tutup
dengan rekomendasi perbaikan spesifik.

\begin{center}\rule{0.5\linewidth}{0.5pt}\end{center}

\section{Rekap Skor (ringkas)}\label{rekap-skor-ringkas}

\begin{itemize}
\tightlist
\item
  \textbf{UTS-1:} 10/20 → \textbf{50\%}
\item
  \textbf{UTS-2:} 7/20 → \textbf{35\%}
\item
  \textbf{UTS-3:} 19/20 → \textbf{95\%}
\item
  \textbf{UTS-4:} 4/20 → \textbf{20\%}
\item
  \textbf{UTS-5:} 7/25 → \textbf{28\%}
\end{itemize}

CSV lengkap sudah saya siapkan untuk dokumentasi dan olah lanjut:
\href{sandbox:/mnt/data/UTS_self_assessment.csv}{Download CSV
ringkasan}.

\section{Langkah Perbaikan Cepat (prioritas 1
minggu)}\label{langkah-perbaikan-cepat-prioritas-1-minggu}

\begin{enumerate}
\def\labelenumi{\arabic{enumi}.}
\tightlist
\item
  \textbf{Lengkapi UTS-4 (My SHAPE)} sesuai saran di atas;
\item
  \textbf{Tambahkan lirik + sinopsis} untuk tiap lagu di UTS-2 agar bisa
  dinilai isi/inspirasinya;
\item
  \textbf{Tulis 1 review personal penuh} (UTS-5) memakai rubrik (pilih
  salah satu karya Anda sendiri sebagai objek);
\item
  Revisi UTS-1 dengan pembuka anekdot + refleksi nilai diri.
\end{enumerate}

Jika Anda mau, saya bisa bantu merapikan UTS-4 (tabel SHAPE + narasi)
dan membuat kerangka cepat untuk \textbf{review} di UTS-5 dari salah
satu karya Anda.

\bookmarksetup{startatroot}

\chapter{UAS-1 My Concepts}\label{uas-1-my-concepts}

Mau hidup epik ? \href{lifestory.pdf}{Write your Life Story}

Apa itu berkonsep?

\url{https://youtu.be/QVfUlVBO80U?si=yM6q_rwV9rcDBbu7}

\bookmarksetup{startatroot}

\chapter{UAS-3 My Opinions}\label{uas-3-my-opinions}

SApa itu beropini? \href{BM\%20Opini.mp4}{Opini Berpengaruh}

Bagiamana menjaadi menarik? \href{./Interesting.mp4}{Menjadi Menarik}

\bookmarksetup{startatroot}

\chapter{UAS-3 My Innovations}\label{uas-3-my-innovations}

\bookmarksetup{startatroot}

\chapter{UAS-4 My Knowledge}\label{uas-4-my-knowledge}

Cara saya mengkomunikasikan sebuah pengatahuan sebagai petunjuk bagi
orang lain 1) saya tulis
\href{Rekomendasi\%20Presentasi\%20Efektif(Contoh\%20Makalah).pdf}{makalah
sebagai bahan utama} 2) lalu saya buat
\href{Contoh\%20Transkrip\%20Presentasi.pdf}{transkrip ucapan lisan} 3)
kemudian saya kembangkan
\href{Rekomendasi\%20Presentasi\%20(Contoh\%20Slides).pdf}{slide
pendukung trnsskrip} 4) lalu saya memproduksivideo audio visual
\url{https://youtu.be/ZbghfMvnPZc} \url{https://youtu.be/ZbghfMvnPZc}

\bookmarksetup{startatroot}

\chapter{UAS-5 My Professional
Reviews}\label{uas-5-my-professional-reviews}

Untuk melAkukan review, seperti pada
\href{../My_Personal_Reviews/Doc.5.Mengevaluasi-Esai-Berdasarkan-Rubrik.pdf}{pendekatan
AI}, kita membutuhkan rubrik

\bookmarksetup{startatroot}

\chapter{Summary}\label{summary}

In summary, this book has no content whatsoever.

\bookmarksetup{startatroot}

\chapter*{References}\label{references}
\addcontentsline{toc}{chapter}{References}

\markboth{References}{References}

\phantomsection\label{refs}




\end{document}
